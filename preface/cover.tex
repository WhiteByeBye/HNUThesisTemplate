% !Mode:: "TeX:UTF-8"

\heading{本科生毕业论文}
\title{论文标题}
\author{学生姓名}
\stdnumber{学生学号}
\major{专业班级}
\affil{学院名称}
\teacher{指导老师}
\dean{学院院长}

\untitle{湖~~南~~大~~学}
\declaretitle{学位论文原创性声明}
\declarecontent{
本人郑重声明:所呈交的论文是本人在导师的指导下独立进行研究所取得的研究成果。除了文中特别加以标注引用的内容外,本论文不包含任何其他个人或集体已经发表或撰写的成果作品。对本文的研究做出重要贡献的个人和集体,均已在文中以明确方式标明。本人完全意识到本声明的法律后果由本人承担。
}
\authorizationtitle{学位论文版权使用授权书}
\authorizationcontent{
本学位论文作者完全了解学校有关保留、使用学位论文的规定,同意学校保留并向国家有关部门或机构送交论文的复印件和电子版,允许论文被查阅和借阅。本人授权湖南大学可以将本学位论文的全部或部分内容编入有关数据库进行检索,可以采用影印、缩印或扫描等复制手段保存和汇编本学位论文。
}
\authorizationadd{本学位论文属于}
\authorsigncap{作者签名:}
\supervisorsigncap{导师签名:}
\signdatecap{签字日期:}

\date{\the\year 年\the\month 月 \the\day 日}

\cabstract{
摘要是论文内容的简要陈述,是一篇具有独立性和完整性的短文。摘要应包括本论文的创造性成果及其理论与实际意义。摘要中不宜使用公式、图表,不标注引用文献编号。避免将摘要写成目录式的内容介绍。
}

\ckeywords{关键词1;~~关键词2;~~关键词3;~~关键词4}

\eabstract{
An abstract is a brief summary of a research article, thesis, review, conference proceeding or any in-depth analysis of a particular subject and is often used to help the reader quickly ascertain the paper's purpose. When used, an abstract always appears at the beginning of a manuscript or typescript, acting as the point-of-entry for any given academic paper or patent application. Abstracting and indexing services for various academic disciplines are aimed at compiling a body of literature for that particular subject.
}

\ekeywords{Key Word1;~~Key Word 2;~~Key Word 3;~~Key Word 4}

\makecover

\clearpage
